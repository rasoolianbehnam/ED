
% Default to the notebook output style

    


% Inherit from the specified cell style.




    
\documentclass[11pt]{article}

    
    
    \usepackage[T1]{fontenc}
    % Nicer default font (+ math font) than Computer Modern for most use cases
    \usepackage{mathpazo}

    % Basic figure setup, for now with no caption control since it's done
    % automatically by Pandoc (which extracts ![](path) syntax from Markdown).
    \usepackage{graphicx}
    % We will generate all images so they have a width \maxwidth. This means
    % that they will get their normal width if they fit onto the page, but
    % are scaled down if they would overflow the margins.
    \makeatletter
    \def\maxwidth{\ifdim\Gin@nat@width>\linewidth\linewidth
    \else\Gin@nat@width\fi}
    \makeatother
    \let\Oldincludegraphics\includegraphics
    % Set max figure width to be 80% of text width, for now hardcoded.
    \renewcommand{\includegraphics}[1]{\Oldincludegraphics[width=.8\maxwidth]{#1}}
    % Ensure that by default, figures have no caption (until we provide a
    % proper Figure object with a Caption API and a way to capture that
    % in the conversion process - todo).
    \usepackage{caption}
    \DeclareCaptionLabelFormat{nolabel}{}
    \captionsetup{labelformat=nolabel}

    \usepackage{adjustbox} % Used to constrain images to a maximum size 
    \usepackage{xcolor} % Allow colors to be defined
    \usepackage{enumerate} % Needed for markdown enumerations to work
    \usepackage{geometry} % Used to adjust the document margins
    \usepackage{amsmath} % Equations
    \usepackage{amssymb} % Equations
    \usepackage{textcomp} % defines textquotesingle
    % Hack from http://tex.stackexchange.com/a/47451/13684:
    \AtBeginDocument{%
        \def\PYZsq{\textquotesingle}% Upright quotes in Pygmentized code
    }
    \usepackage{upquote} % Upright quotes for verbatim code
    \usepackage{eurosym} % defines \euro
    \usepackage[mathletters]{ucs} % Extended unicode (utf-8) support
    \usepackage[utf8x]{inputenc} % Allow utf-8 characters in the tex document
    \usepackage{fancyvrb} % verbatim replacement that allows latex
    \usepackage{grffile} % extends the file name processing of package graphics 
                         % to support a larger range 
    % The hyperref package gives us a pdf with properly built
    % internal navigation ('pdf bookmarks' for the table of contents,
    % internal cross-reference links, web links for URLs, etc.)
    \usepackage{hyperref}
    \usepackage{longtable} % longtable support required by pandoc >1.10
    \usepackage{booktabs}  % table support for pandoc > 1.12.2
    \usepackage[inline]{enumitem} % IRkernel/repr support (it uses the enumerate* environment)
    \usepackage[normalem]{ulem} % ulem is needed to support strikethroughs (\sout)
                                % normalem makes italics be italics, not underlines
    

    
    
    % Colors for the hyperref package
    \definecolor{urlcolor}{rgb}{0,.145,.698}
    \definecolor{linkcolor}{rgb}{.71,0.21,0.01}
    \definecolor{citecolor}{rgb}{.12,.54,.11}

    % ANSI colors
    \definecolor{ansi-black}{HTML}{3E424D}
    \definecolor{ansi-black-intense}{HTML}{282C36}
    \definecolor{ansi-red}{HTML}{E75C58}
    \definecolor{ansi-red-intense}{HTML}{B22B31}
    \definecolor{ansi-green}{HTML}{00A250}
    \definecolor{ansi-green-intense}{HTML}{007427}
    \definecolor{ansi-yellow}{HTML}{DDB62B}
    \definecolor{ansi-yellow-intense}{HTML}{B27D12}
    \definecolor{ansi-blue}{HTML}{208FFB}
    \definecolor{ansi-blue-intense}{HTML}{0065CA}
    \definecolor{ansi-magenta}{HTML}{D160C4}
    \definecolor{ansi-magenta-intense}{HTML}{A03196}
    \definecolor{ansi-cyan}{HTML}{60C6C8}
    \definecolor{ansi-cyan-intense}{HTML}{258F8F}
    \definecolor{ansi-white}{HTML}{C5C1B4}
    \definecolor{ansi-white-intense}{HTML}{A1A6B2}

    % commands and environments needed by pandoc snippets
    % extracted from the output of `pandoc -s`
    \providecommand{\tightlist}{%
      \setlength{\itemsep}{0pt}\setlength{\parskip}{0pt}}
    \DefineVerbatimEnvironment{Highlighting}{Verbatim}{commandchars=\\\{\}}
    % Add ',fontsize=\small' for more characters per line
    \newenvironment{Shaded}{}{}
    \newcommand{\KeywordTok}[1]{\textcolor[rgb]{0.00,0.44,0.13}{\textbf{{#1}}}}
    \newcommand{\DataTypeTok}[1]{\textcolor[rgb]{0.56,0.13,0.00}{{#1}}}
    \newcommand{\DecValTok}[1]{\textcolor[rgb]{0.25,0.63,0.44}{{#1}}}
    \newcommand{\BaseNTok}[1]{\textcolor[rgb]{0.25,0.63,0.44}{{#1}}}
    \newcommand{\FloatTok}[1]{\textcolor[rgb]{0.25,0.63,0.44}{{#1}}}
    \newcommand{\CharTok}[1]{\textcolor[rgb]{0.25,0.44,0.63}{{#1}}}
    \newcommand{\StringTok}[1]{\textcolor[rgb]{0.25,0.44,0.63}{{#1}}}
    \newcommand{\CommentTok}[1]{\textcolor[rgb]{0.38,0.63,0.69}{\textit{{#1}}}}
    \newcommand{\OtherTok}[1]{\textcolor[rgb]{0.00,0.44,0.13}{{#1}}}
    \newcommand{\AlertTok}[1]{\textcolor[rgb]{1.00,0.00,0.00}{\textbf{{#1}}}}
    \newcommand{\FunctionTok}[1]{\textcolor[rgb]{0.02,0.16,0.49}{{#1}}}
    \newcommand{\RegionMarkerTok}[1]{{#1}}
    \newcommand{\ErrorTok}[1]{\textcolor[rgb]{1.00,0.00,0.00}{\textbf{{#1}}}}
    \newcommand{\NormalTok}[1]{{#1}}
    
    % Additional commands for more recent versions of Pandoc
    \newcommand{\ConstantTok}[1]{\textcolor[rgb]{0.53,0.00,0.00}{{#1}}}
    \newcommand{\SpecialCharTok}[1]{\textcolor[rgb]{0.25,0.44,0.63}{{#1}}}
    \newcommand{\VerbatimStringTok}[1]{\textcolor[rgb]{0.25,0.44,0.63}{{#1}}}
    \newcommand{\SpecialStringTok}[1]{\textcolor[rgb]{0.73,0.40,0.53}{{#1}}}
    \newcommand{\ImportTok}[1]{{#1}}
    \newcommand{\DocumentationTok}[1]{\textcolor[rgb]{0.73,0.13,0.13}{\textit{{#1}}}}
    \newcommand{\AnnotationTok}[1]{\textcolor[rgb]{0.38,0.63,0.69}{\textbf{\textit{{#1}}}}}
    \newcommand{\CommentVarTok}[1]{\textcolor[rgb]{0.38,0.63,0.69}{\textbf{\textit{{#1}}}}}
    \newcommand{\VariableTok}[1]{\textcolor[rgb]{0.10,0.09,0.49}{{#1}}}
    \newcommand{\ControlFlowTok}[1]{\textcolor[rgb]{0.00,0.44,0.13}{\textbf{{#1}}}}
    \newcommand{\OperatorTok}[1]{\textcolor[rgb]{0.40,0.40,0.40}{{#1}}}
    \newcommand{\BuiltInTok}[1]{{#1}}
    \newcommand{\ExtensionTok}[1]{{#1}}
    \newcommand{\PreprocessorTok}[1]{\textcolor[rgb]{0.74,0.48,0.00}{{#1}}}
    \newcommand{\AttributeTok}[1]{\textcolor[rgb]{0.49,0.56,0.16}{{#1}}}
    \newcommand{\InformationTok}[1]{\textcolor[rgb]{0.38,0.63,0.69}{\textbf{\textit{{#1}}}}}
    \newcommand{\WarningTok}[1]{\textcolor[rgb]{0.38,0.63,0.69}{\textbf{\textit{{#1}}}}}
    
    
    % Define a nice break command that doesn't care if a line doesn't already
    % exist.
    \def\br{\hspace*{\fill} \\* }
    % Math Jax compatability definitions
    \def\gt{>}
    \def\lt{<}
    % Document parameters
    \title{grading}
    
    
    

    % Pygments definitions
    
\makeatletter
\def\PY@reset{\let\PY@it=\relax \let\PY@bf=\relax%
    \let\PY@ul=\relax \let\PY@tc=\relax%
    \let\PY@bc=\relax \let\PY@ff=\relax}
\def\PY@tok#1{\csname PY@tok@#1\endcsname}
\def\PY@toks#1+{\ifx\relax#1\empty\else%
    \PY@tok{#1}\expandafter\PY@toks\fi}
\def\PY@do#1{\PY@bc{\PY@tc{\PY@ul{%
    \PY@it{\PY@bf{\PY@ff{#1}}}}}}}
\def\PY#1#2{\PY@reset\PY@toks#1+\relax+\PY@do{#2}}

\expandafter\def\csname PY@tok@w\endcsname{\def\PY@tc##1{\textcolor[rgb]{0.73,0.73,0.73}{##1}}}
\expandafter\def\csname PY@tok@c\endcsname{\let\PY@it=\textit\def\PY@tc##1{\textcolor[rgb]{0.25,0.50,0.50}{##1}}}
\expandafter\def\csname PY@tok@cp\endcsname{\def\PY@tc##1{\textcolor[rgb]{0.74,0.48,0.00}{##1}}}
\expandafter\def\csname PY@tok@k\endcsname{\let\PY@bf=\textbf\def\PY@tc##1{\textcolor[rgb]{0.00,0.50,0.00}{##1}}}
\expandafter\def\csname PY@tok@kp\endcsname{\def\PY@tc##1{\textcolor[rgb]{0.00,0.50,0.00}{##1}}}
\expandafter\def\csname PY@tok@kt\endcsname{\def\PY@tc##1{\textcolor[rgb]{0.69,0.00,0.25}{##1}}}
\expandafter\def\csname PY@tok@o\endcsname{\def\PY@tc##1{\textcolor[rgb]{0.40,0.40,0.40}{##1}}}
\expandafter\def\csname PY@tok@ow\endcsname{\let\PY@bf=\textbf\def\PY@tc##1{\textcolor[rgb]{0.67,0.13,1.00}{##1}}}
\expandafter\def\csname PY@tok@nb\endcsname{\def\PY@tc##1{\textcolor[rgb]{0.00,0.50,0.00}{##1}}}
\expandafter\def\csname PY@tok@nf\endcsname{\def\PY@tc##1{\textcolor[rgb]{0.00,0.00,1.00}{##1}}}
\expandafter\def\csname PY@tok@nc\endcsname{\let\PY@bf=\textbf\def\PY@tc##1{\textcolor[rgb]{0.00,0.00,1.00}{##1}}}
\expandafter\def\csname PY@tok@nn\endcsname{\let\PY@bf=\textbf\def\PY@tc##1{\textcolor[rgb]{0.00,0.00,1.00}{##1}}}
\expandafter\def\csname PY@tok@ne\endcsname{\let\PY@bf=\textbf\def\PY@tc##1{\textcolor[rgb]{0.82,0.25,0.23}{##1}}}
\expandafter\def\csname PY@tok@nv\endcsname{\def\PY@tc##1{\textcolor[rgb]{0.10,0.09,0.49}{##1}}}
\expandafter\def\csname PY@tok@no\endcsname{\def\PY@tc##1{\textcolor[rgb]{0.53,0.00,0.00}{##1}}}
\expandafter\def\csname PY@tok@nl\endcsname{\def\PY@tc##1{\textcolor[rgb]{0.63,0.63,0.00}{##1}}}
\expandafter\def\csname PY@tok@ni\endcsname{\let\PY@bf=\textbf\def\PY@tc##1{\textcolor[rgb]{0.60,0.60,0.60}{##1}}}
\expandafter\def\csname PY@tok@na\endcsname{\def\PY@tc##1{\textcolor[rgb]{0.49,0.56,0.16}{##1}}}
\expandafter\def\csname PY@tok@nt\endcsname{\let\PY@bf=\textbf\def\PY@tc##1{\textcolor[rgb]{0.00,0.50,0.00}{##1}}}
\expandafter\def\csname PY@tok@nd\endcsname{\def\PY@tc##1{\textcolor[rgb]{0.67,0.13,1.00}{##1}}}
\expandafter\def\csname PY@tok@s\endcsname{\def\PY@tc##1{\textcolor[rgb]{0.73,0.13,0.13}{##1}}}
\expandafter\def\csname PY@tok@sd\endcsname{\let\PY@it=\textit\def\PY@tc##1{\textcolor[rgb]{0.73,0.13,0.13}{##1}}}
\expandafter\def\csname PY@tok@si\endcsname{\let\PY@bf=\textbf\def\PY@tc##1{\textcolor[rgb]{0.73,0.40,0.53}{##1}}}
\expandafter\def\csname PY@tok@se\endcsname{\let\PY@bf=\textbf\def\PY@tc##1{\textcolor[rgb]{0.73,0.40,0.13}{##1}}}
\expandafter\def\csname PY@tok@sr\endcsname{\def\PY@tc##1{\textcolor[rgb]{0.73,0.40,0.53}{##1}}}
\expandafter\def\csname PY@tok@ss\endcsname{\def\PY@tc##1{\textcolor[rgb]{0.10,0.09,0.49}{##1}}}
\expandafter\def\csname PY@tok@sx\endcsname{\def\PY@tc##1{\textcolor[rgb]{0.00,0.50,0.00}{##1}}}
\expandafter\def\csname PY@tok@m\endcsname{\def\PY@tc##1{\textcolor[rgb]{0.40,0.40,0.40}{##1}}}
\expandafter\def\csname PY@tok@gh\endcsname{\let\PY@bf=\textbf\def\PY@tc##1{\textcolor[rgb]{0.00,0.00,0.50}{##1}}}
\expandafter\def\csname PY@tok@gu\endcsname{\let\PY@bf=\textbf\def\PY@tc##1{\textcolor[rgb]{0.50,0.00,0.50}{##1}}}
\expandafter\def\csname PY@tok@gd\endcsname{\def\PY@tc##1{\textcolor[rgb]{0.63,0.00,0.00}{##1}}}
\expandafter\def\csname PY@tok@gi\endcsname{\def\PY@tc##1{\textcolor[rgb]{0.00,0.63,0.00}{##1}}}
\expandafter\def\csname PY@tok@gr\endcsname{\def\PY@tc##1{\textcolor[rgb]{1.00,0.00,0.00}{##1}}}
\expandafter\def\csname PY@tok@ge\endcsname{\let\PY@it=\textit}
\expandafter\def\csname PY@tok@gs\endcsname{\let\PY@bf=\textbf}
\expandafter\def\csname PY@tok@gp\endcsname{\let\PY@bf=\textbf\def\PY@tc##1{\textcolor[rgb]{0.00,0.00,0.50}{##1}}}
\expandafter\def\csname PY@tok@go\endcsname{\def\PY@tc##1{\textcolor[rgb]{0.53,0.53,0.53}{##1}}}
\expandafter\def\csname PY@tok@gt\endcsname{\def\PY@tc##1{\textcolor[rgb]{0.00,0.27,0.87}{##1}}}
\expandafter\def\csname PY@tok@err\endcsname{\def\PY@bc##1{\setlength{\fboxsep}{0pt}\fcolorbox[rgb]{1.00,0.00,0.00}{1,1,1}{\strut ##1}}}
\expandafter\def\csname PY@tok@kc\endcsname{\let\PY@bf=\textbf\def\PY@tc##1{\textcolor[rgb]{0.00,0.50,0.00}{##1}}}
\expandafter\def\csname PY@tok@kd\endcsname{\let\PY@bf=\textbf\def\PY@tc##1{\textcolor[rgb]{0.00,0.50,0.00}{##1}}}
\expandafter\def\csname PY@tok@kn\endcsname{\let\PY@bf=\textbf\def\PY@tc##1{\textcolor[rgb]{0.00,0.50,0.00}{##1}}}
\expandafter\def\csname PY@tok@kr\endcsname{\let\PY@bf=\textbf\def\PY@tc##1{\textcolor[rgb]{0.00,0.50,0.00}{##1}}}
\expandafter\def\csname PY@tok@bp\endcsname{\def\PY@tc##1{\textcolor[rgb]{0.00,0.50,0.00}{##1}}}
\expandafter\def\csname PY@tok@fm\endcsname{\def\PY@tc##1{\textcolor[rgb]{0.00,0.00,1.00}{##1}}}
\expandafter\def\csname PY@tok@vc\endcsname{\def\PY@tc##1{\textcolor[rgb]{0.10,0.09,0.49}{##1}}}
\expandafter\def\csname PY@tok@vg\endcsname{\def\PY@tc##1{\textcolor[rgb]{0.10,0.09,0.49}{##1}}}
\expandafter\def\csname PY@tok@vi\endcsname{\def\PY@tc##1{\textcolor[rgb]{0.10,0.09,0.49}{##1}}}
\expandafter\def\csname PY@tok@vm\endcsname{\def\PY@tc##1{\textcolor[rgb]{0.10,0.09,0.49}{##1}}}
\expandafter\def\csname PY@tok@sa\endcsname{\def\PY@tc##1{\textcolor[rgb]{0.73,0.13,0.13}{##1}}}
\expandafter\def\csname PY@tok@sb\endcsname{\def\PY@tc##1{\textcolor[rgb]{0.73,0.13,0.13}{##1}}}
\expandafter\def\csname PY@tok@sc\endcsname{\def\PY@tc##1{\textcolor[rgb]{0.73,0.13,0.13}{##1}}}
\expandafter\def\csname PY@tok@dl\endcsname{\def\PY@tc##1{\textcolor[rgb]{0.73,0.13,0.13}{##1}}}
\expandafter\def\csname PY@tok@s2\endcsname{\def\PY@tc##1{\textcolor[rgb]{0.73,0.13,0.13}{##1}}}
\expandafter\def\csname PY@tok@sh\endcsname{\def\PY@tc##1{\textcolor[rgb]{0.73,0.13,0.13}{##1}}}
\expandafter\def\csname PY@tok@s1\endcsname{\def\PY@tc##1{\textcolor[rgb]{0.73,0.13,0.13}{##1}}}
\expandafter\def\csname PY@tok@mb\endcsname{\def\PY@tc##1{\textcolor[rgb]{0.40,0.40,0.40}{##1}}}
\expandafter\def\csname PY@tok@mf\endcsname{\def\PY@tc##1{\textcolor[rgb]{0.40,0.40,0.40}{##1}}}
\expandafter\def\csname PY@tok@mh\endcsname{\def\PY@tc##1{\textcolor[rgb]{0.40,0.40,0.40}{##1}}}
\expandafter\def\csname PY@tok@mi\endcsname{\def\PY@tc##1{\textcolor[rgb]{0.40,0.40,0.40}{##1}}}
\expandafter\def\csname PY@tok@il\endcsname{\def\PY@tc##1{\textcolor[rgb]{0.40,0.40,0.40}{##1}}}
\expandafter\def\csname PY@tok@mo\endcsname{\def\PY@tc##1{\textcolor[rgb]{0.40,0.40,0.40}{##1}}}
\expandafter\def\csname PY@tok@ch\endcsname{\let\PY@it=\textit\def\PY@tc##1{\textcolor[rgb]{0.25,0.50,0.50}{##1}}}
\expandafter\def\csname PY@tok@cm\endcsname{\let\PY@it=\textit\def\PY@tc##1{\textcolor[rgb]{0.25,0.50,0.50}{##1}}}
\expandafter\def\csname PY@tok@cpf\endcsname{\let\PY@it=\textit\def\PY@tc##1{\textcolor[rgb]{0.25,0.50,0.50}{##1}}}
\expandafter\def\csname PY@tok@c1\endcsname{\let\PY@it=\textit\def\PY@tc##1{\textcolor[rgb]{0.25,0.50,0.50}{##1}}}
\expandafter\def\csname PY@tok@cs\endcsname{\let\PY@it=\textit\def\PY@tc##1{\textcolor[rgb]{0.25,0.50,0.50}{##1}}}

\def\PYZbs{\char`\\}
\def\PYZus{\char`\_}
\def\PYZob{\char`\{}
\def\PYZcb{\char`\}}
\def\PYZca{\char`\^}
\def\PYZam{\char`\&}
\def\PYZlt{\char`\<}
\def\PYZgt{\char`\>}
\def\PYZsh{\char`\#}
\def\PYZpc{\char`\%}
\def\PYZdl{\char`\$}
\def\PYZhy{\char`\-}
\def\PYZsq{\char`\'}
\def\PYZdq{\char`\"}
\def\PYZti{\char`\~}
% for compatibility with earlier versions
\def\PYZat{@}
\def\PYZlb{[}
\def\PYZrb{]}
\makeatother


    % Exact colors from NB
    \definecolor{incolor}{rgb}{0.0, 0.0, 0.5}
    \definecolor{outcolor}{rgb}{0.545, 0.0, 0.0}



    
    % Prevent overflowing lines due to hard-to-break entities
    \sloppy 
    % Setup hyperref package
    \hypersetup{
      breaklinks=true,  % so long urls are correctly broken across lines
      colorlinks=true,
      urlcolor=urlcolor,
      linkcolor=linkcolor,
      citecolor=citecolor,
      }
    % Slightly bigger margins than the latex defaults
    
    \geometry{verbose,tmargin=1in,bmargin=1in,lmargin=1in,rmargin=1in}
    
    

    \begin{document}
    
    
    \maketitle
    
    

    
    \begin{Verbatim}[commandchars=\\\{\}]
{\color{incolor}In [{\color{incolor}223}]:} \PY{k+kn}{from} \PY{n+nn}{IPython}\PY{n+nn}{.}\PY{n+nn}{display} \PY{k}{import} \PY{n}{display}\PY{p}{,} \PY{n}{Math}\PY{p}{,} \PY{n}{Latex}\PY{p}{,} \PY{n}{Markdown}
\end{Verbatim}


    \begin{Verbatim}[commandchars=\\\{\}]
{\color{incolor}In [{\color{incolor}224}]:} \PY{n}{ps} \PY{o}{=} \PY{p}{\PYZob{}}\PY{p}{\PYZcb{}}
\end{Verbatim}


    \begin{Verbatim}[commandchars=\\\{\}]
{\color{incolor}In [{\color{incolor}225}]:} \PY{n}{text} \PY{o}{=} \PY{n}{ps}\PY{p}{[}\PY{l+m+mf}{1.08}\PY{p}{]} \PY{o}{=} \PY{l+s+sa}{r}\PY{l+s+s1}{\PYZsq{}}\PY{l+s+s1}{\PYZdl{}m = 2k\PYZus{}1 + 1, n = 2k\PYZus{}2 + 1 }\PY{l+s+s1}{\PYZbs{}}\PY{l+s+s1}{Rightarrow m + n = 2k\PYZus{}1 + 2\PYZus{}k2 + 2 = 2(k\PYZus{}1+k\PYZus{}2+1) = 2k\PYZdl{}}\PY{l+s+s1}{\PYZsq{}}
          \PY{n}{display}\PY{p}{(}\PY{n}{Markdown}\PY{p}{(}\PY{n}{text}\PY{p}{)}\PY{p}{)}
\end{Verbatim}


    \(m = 2k_1 + 1, n = 2k_2 + 1 \Rightarrow m + n = 2k_1 + 2_k2 + 2 = 2(k_1+k_2+1) = 2k\)

    
    \begin{Verbatim}[commandchars=\\\{\}]
{\color{incolor}In [{\color{incolor}226}]:} \PY{n}{text} \PY{o}{=} \PY{n}{ps}\PY{p}{[}\PY{l+m+mf}{1.11}\PY{p}{]} \PY{o}{=} \PY{l+s+sa}{r}\PY{l+s+s2}{\PYZdq{}\PYZdq{}\PYZdq{}}\PY{l+s+s2}{\PYZdl{}m = 2k\PYZus{}1 + 1, n = 2k\PYZus{}2 }\PY{l+s+s2}{\PYZbs{}}\PY{l+s+s2}{Rightarrow m  }\PY{l+s+s2}{\PYZbs{}}\PY{l+s+s2}{times n = (2k\PYZus{}1 + 1)}\PY{l+s+s2}{\PYZbs{}}\PY{l+s+s2}{times (2\PYZus{}k2) = 2 }\PY{l+s+s2}{\PYZbs{}}\PY{l+s+s2}{times k2(2k\PYZus{}1+1) = 2k\PYZdl{}}\PY{l+s+s2}{\PYZdq{}\PYZdq{}\PYZdq{}}
          \PY{n}{display}\PY{p}{(}\PY{n}{Markdown}\PY{p}{(}\PY{n}{text}\PY{p}{)}\PY{p}{)}
\end{Verbatim}


    \(m = 2k_1 + 1, n = 2k_2 \Rightarrow m \times n = (2k_1 + 1)\times (2_k2) = 2 \times k2(2k_1+1) = 2k\)

    
    \begin{Verbatim}[commandchars=\\\{\}]
{\color{incolor}In [{\color{incolor}227}]:} \PY{n}{text} \PY{o}{=} \PY{n}{ps}\PY{p}{[}\PY{l+m+mf}{1.20}\PY{p}{]} \PY{o}{=} \PY{l+s+sa}{r}\PY{l+s+s2}{\PYZdq{}\PYZdq{}\PYZdq{}}\PY{l+s+s2}{\PYZdl{}x }\PY{l+s+s2}{\PYZbs{}}\PY{l+s+s2}{in X  }\PY{l+s+s2}{\PYZbs{}}\PY{l+s+s2}{rightarrow x }\PY{l+s+s2}{\PYZbs{}}\PY{l+s+s2}{in X }\PY{l+s+s2}{\PYZbs{}}\PY{l+s+s2}{vee x }\PY{l+s+s2}{\PYZbs{}}\PY{l+s+s2}{in Y }\PY{l+s+s2}{\PYZbs{}}\PY{l+s+s2}{rightarrow x }\PY{l+s+s2}{\PYZbs{}}\PY{l+s+s2}{in X }\PY{l+s+s2}{\PYZbs{}}\PY{l+s+s2}{cup y }\PY{l+s+s2}{\PYZbs{}}\PY{l+s+s2}{rightarrow X }\PY{l+s+s2}{\PYZbs{}}\PY{l+s+s2}{subseteq X }\PY{l+s+s2}{\PYZbs{}}\PY{l+s+s2}{cup Y\PYZdl{}}\PY{l+s+s2}{\PYZdq{}\PYZdq{}\PYZdq{}}
          \PY{n}{display}\PY{p}{(}\PY{n}{Markdown}\PY{p}{(}\PY{n}{text}\PY{p}{)}\PY{p}{)}
\end{Verbatim}


    \(x \in X \rightarrow x \in X \vee x \in Y \rightarrow x \in X \cup y \rightarrow X \subseteq X \cup Y\)

    
    \begin{Verbatim}[commandchars=\\\{\}]
{\color{incolor}In [{\color{incolor}228}]:} \PY{n}{text} \PY{o}{=} \PY{n}{ps}\PY{p}{[}\PY{l+m+mf}{1.26}\PY{p}{]}\PY{o}{=} \PY{l+s+sa}{r}\PY{l+s+s2}{\PYZdq{}\PYZdq{}\PYZdq{}}
          \PY{l+s+s2}{\PYZdl{}A }\PY{l+s+s2}{\PYZbs{}}\PY{l+s+s2}{in }\PY{l+s+s2}{\PYZbs{}}\PY{l+s+s2}{wp(X) }\PY{l+s+s2}{\PYZbs{}}\PY{l+s+s2}{cup }\PY{l+s+s2}{\PYZbs{}}\PY{l+s+s2}{wp(Y) \PYZhy{}\PYZgt{} A }\PY{l+s+s2}{\PYZbs{}}\PY{l+s+s2}{in }\PY{l+s+s2}{\PYZbs{}}\PY{l+s+s2}{wp(X) }\PY{l+s+s2}{\PYZbs{}}\PY{l+s+s2}{vee A }\PY{l+s+s2}{\PYZbs{}}\PY{l+s+s2}{in }\PY{l+s+s2}{\PYZbs{}}\PY{l+s+s2}{wp(Y) }
          \PY{l+s+s2}{\PYZbs{}}\PY{l+s+s2}{rightarrow A }\PY{l+s+s2}{\PYZbs{}}\PY{l+s+s2}{subseteq X }\PY{l+s+s2}{\PYZbs{}}\PY{l+s+s2}{vee A }\PY{l+s+s2}{\PYZbs{}}\PY{l+s+s2}{subseteq Y }\PY{l+s+s2}{\PYZbs{}}\PY{l+s+s2}{rightarrow A }\PY{l+s+s2}{\PYZbs{}}\PY{l+s+s2}{subseteq X }\PY{l+s+s2}{\PYZbs{}}\PY{l+s+s2}{cup Y }\PY{l+s+s2}{\PYZbs{}}\PY{l+s+s2}{rightarrow A }\PY{l+s+s2}{\PYZbs{}}\PY{l+s+s2}{in }\PY{l+s+s2}{\PYZbs{}}\PY{l+s+s2}{wp(X }\PY{l+s+s2}{\PYZbs{}}\PY{l+s+s2}{cup Y)\PYZdl{}}
          \PY{l+s+s2}{\PYZdq{}\PYZdq{}\PYZdq{}}
          \PY{n}{display}\PY{p}{(}\PY{n}{Markdown}\PY{p}{(}\PY{n}{text}\PY{p}{)}\PY{p}{)}
\end{Verbatim}


    \(A \in \wp(X) \cup \wp(Y) -> A \in \wp(X) \vee A \in \wp(Y) \rightarrow A \subseteq X \vee A \subseteq Y \rightarrow A \subseteq X \cup Y \rightarrow A \in \wp(X \cup Y)\)

    
    \begin{Verbatim}[commandchars=\\\{\}]
{\color{incolor}In [{\color{incolor}229}]:} \PY{n}{text} \PY{o}{=} \PY{n}{ps}\PY{p}{[}\PY{l+m+mf}{1.45}\PY{p}{]}\PY{o}{=} \PY{l+s+sa}{r}\PY{l+s+s2}{\PYZdq{}\PYZdq{}\PYZdq{}}
          \PY{l+s+s2}{(1): \PYZdl{}a }\PY{l+s+s2}{\PYZbs{}}\PY{l+s+s2}{in X }\PY{l+s+s2}{\PYZbs{}}\PY{l+s+s2}{cup Y \PYZhy{}\PYZgt{} a }\PY{l+s+s2}{\PYZbs{}}\PY{l+s+s2}{in X }\PY{l+s+s2}{\PYZbs{}}\PY{l+s+s2}{vee a }\PY{l+s+s2}{\PYZbs{}}\PY{l+s+s2}{in Y \PYZhy{}\PYZgt{} a }\PY{l+s+s2}{\PYZbs{}}\PY{l+s+s2}{in Y }\PY{l+s+s2}{\PYZbs{}}\PY{l+s+s2}{vee a }\PY{l+s+s2}{\PYZbs{}}\PY{l+s+s2}{in X \PYZhy{}\PYZgt{} a }\PY{l+s+s2}{\PYZbs{}}\PY{l+s+s2}{in Y }\PY{l+s+s2}{\PYZbs{}}\PY{l+s+s2}{cup X \PYZhy{}\PYZgt{} X }\PY{l+s+s2}{\PYZbs{}}\PY{l+s+s2}{cup Y }\PY{l+s+s2}{\PYZbs{}}\PY{l+s+s2}{subseteq Y }\PY{l+s+s2}{\PYZbs{}}\PY{l+s+s2}{cup X\PYZdl{}}
          
          \PY{l+s+s2}{(2): \PYZdl{}a }\PY{l+s+s2}{\PYZbs{}}\PY{l+s+s2}{in Y }\PY{l+s+s2}{\PYZbs{}}\PY{l+s+s2}{cup X \PYZhy{}\PYZgt{} a }\PY{l+s+s2}{\PYZbs{}}\PY{l+s+s2}{in Y }\PY{l+s+s2}{\PYZbs{}}\PY{l+s+s2}{vee a }\PY{l+s+s2}{\PYZbs{}}\PY{l+s+s2}{in X \PYZhy{}\PYZgt{} a }\PY{l+s+s2}{\PYZbs{}}\PY{l+s+s2}{in X }\PY{l+s+s2}{\PYZbs{}}\PY{l+s+s2}{vee a }\PY{l+s+s2}{\PYZbs{}}\PY{l+s+s2}{in Y \PYZhy{}\PYZgt{} a }\PY{l+s+s2}{\PYZbs{}}\PY{l+s+s2}{in X }\PY{l+s+s2}{\PYZbs{}}\PY{l+s+s2}{cup Y \PYZhy{}\PYZgt{} Y }\PY{l+s+s2}{\PYZbs{}}\PY{l+s+s2}{cup X }\PY{l+s+s2}{\PYZbs{}}\PY{l+s+s2}{subseteq X }\PY{l+s+s2}{\PYZbs{}}\PY{l+s+s2}{cup Y\PYZdl{}}
          
          \PY{l+s+s2}{(1) + (2) \PYZdl{}\PYZhy{}\PYZgt{} X }\PY{l+s+s2}{\PYZbs{}}\PY{l+s+s2}{cup Y = Y }\PY{l+s+s2}{\PYZbs{}}\PY{l+s+s2}{cup X\PYZdl{}}
          
          \PY{l+s+s2}{The same goes for \PYZdl{}}\PY{l+s+s2}{\PYZbs{}}\PY{l+s+s2}{cap\PYZdl{}}
          \PY{l+s+s2}{\PYZdq{}\PYZdq{}\PYZdq{}}\PY{o}{.}\PY{n}{replace}\PY{p}{(}\PY{l+s+s1}{\PYZsq{}}\PY{l+s+s1}{\PYZhy{}\PYZgt{}}\PY{l+s+s1}{\PYZsq{}}\PY{p}{,} \PY{l+s+s1}{\PYZsq{}}\PY{l+s+se}{\PYZbs{}\PYZbs{}}\PY{l+s+s1}{rightarrow}\PY{l+s+s1}{\PYZsq{}}\PY{p}{)}
          \PY{n}{display}\PY{p}{(}\PY{n}{Markdown}\PY{p}{(}\PY{n}{text}\PY{p}{)}\PY{p}{)}
\end{Verbatim}


    (1):
\(a \in X \cup Y \rightarrow a \in X \vee a \in Y \rightarrow a \in Y \vee a \in X \rightarrow a \in Y \cup X \rightarrow X \cup Y \subseteq Y \cup X\)

(2):
\(a \in Y \cup X \rightarrow a \in Y \vee a \in X \rightarrow a \in X \vee a \in Y \rightarrow a \in X \cup Y \rightarrow Y \cup X \subseteq X \cup Y\)

\begin{enumerate}
\def\labelenumi{(\arabic{enumi})}
\item
  \begin{itemize}
  \item
    \begin{enumerate}
    \def\labelenumii{(\arabic{enumii})}
    \setcounter{enumii}{1}
    \tightlist
    \item
      \(\rightarrow X \cup Y = Y \cup X\)
    \end{enumerate}
  \end{itemize}
\end{enumerate}

The same goes for \(\cap\)

    
    \begin{Verbatim}[commandchars=\\\{\}]
{\color{incolor}In [{\color{incolor}230}]:} \PY{n}{text} \PY{o}{=} \PY{n}{ps}\PY{p}{[}\PY{l+m+mf}{2.02}\PY{p}{]}\PY{o}{=} \PY{l+s+sa}{r}\PY{l+s+s2}{\PYZdq{}\PYZdq{}\PYZdq{}}
          \PY{l+s+s2}{Converse: for all \PYZdl{}x }\PY{l+s+s2}{\PYZbs{}}\PY{l+s+s2}{in }\PY{l+s+s2}{\PYZbs{}}\PY{l+s+s2}{mathbb}\PY{l+s+si}{\PYZob{}R\PYZcb{}}\PY{l+s+s2}{\PYZdl{} if \PYZdl{}x\PYZdl{} is irrational, then \PYZdl{}x\PYZca{}2\PYZdl{} is irrational.}
          
          \PY{l+s+s2}{Proof by contradiction: \PYZdl{}x = }\PY{l+s+s2}{\PYZbs{}}\PY{l+s+s2}{sqrt(2)\PYZdl{} is irrational}
          \PY{l+s+s2}{but \PYZdl{}x\PYZca{}2 = 2\PYZdl{} is rational.}
          \PY{l+s+s2}{\PYZdq{}\PYZdq{}\PYZdq{}}
          \PY{n}{display}\PY{p}{(}\PY{n}{Markdown}\PY{p}{(}\PY{n}{text}\PY{p}{)}\PY{p}{)}
\end{Verbatim}


    Converse: for all \(x \in \mathbb{R}\) if \(x\) is irrational, then
\(x^2\) is irrational.

Proof by contradiction: \(x = \sqrt(2)\) is irrational but \(x^2 = 2\)
is rational.

    
    \begin{Verbatim}[commandchars=\\\{\}]
{\color{incolor}In [{\color{incolor}231}]:} \PY{n}{text} \PY{o}{=} \PY{n}{ps}\PY{p}{[}\PY{l+m+mf}{2.14}\PY{p}{]}\PY{o}{=} \PY{l+s+sa}{r}\PY{l+s+s2}{\PYZdq{}\PYZdq{}\PYZdq{}}
          \PY{l+s+s2}{Since it says }\PY{l+s+s2}{\PYZsq{}}\PY{l+s+s2}{\PYZsq{}}\PY{l+s+s2}{There exists}\PY{l+s+s2}{\PYZsq{}}\PY{l+s+s2}{\PYZsq{}}\PY{l+s+s2}{, it is sufficient to find a single example that satisfies the }
          \PY{l+s+s2}{argument.}
          
          \PY{l+s+s2}{Example: \PYZdl{}a = 2, b = 1/2 }\PY{l+s+s2}{\PYZbs{}}\PY{l+s+s2}{rightarrow a\PYZdl{}\PYZca{}\PYZdl{}b = a\PYZdl{}\PYZca{}\PYZdl{}(1/2) =  }\PY{l+s+s2}{\PYZbs{}}\PY{l+s+s2}{sqrt}\PY{l+s+si}{\PYZob{}2\PYZcb{}}\PY{l+s+s2}{\PYZdl{}}
          \PY{l+s+s2}{\PYZdq{}\PYZdq{}\PYZdq{}}
          \PY{n}{display}\PY{p}{(}\PY{n}{Markdown}\PY{p}{(}\PY{n}{text}\PY{p}{)}\PY{p}{)}
\end{Verbatim}


    Since it says `'There exists'', it is sufficient to find a single
example that satisfies the argument.

Example:
\(a = 2, b = 1/2 \rightarrow a\)\textsuperscript{\(b = a\)}\((1/2) = \sqrt{2}\)

    
    \begin{Verbatim}[commandchars=\\\{\}]
{\color{incolor}In [{\color{incolor}232}]:} \PY{n}{text} \PY{o}{=} \PY{n}{ps}\PY{p}{[}\PY{l+m+mf}{2.15}\PY{p}{]}\PY{o}{=} \PY{l+s+sa}{r}\PY{l+s+s2}{\PYZdq{}\PYZdq{}\PYZdq{}}
          \PY{l+s+s2}{Proof by contradiction:}
          
          \PY{l+s+s2}{Assume \PYZdl{}x \PYZgt{} y\PYZdl{} then we have:}
          
          \PY{l+s+s2}{\PYZdl{}}\PY{l+s+s2}{\PYZbs{}}\PY{l+s+s2}{exists }\PY{l+s+s2}{\PYZbs{}}\PY{l+s+s2}{epsilon\PYZus{}1 \PYZgt{} 0: x \PYZgt{}= y + }\PY{l+s+s2}{\PYZbs{}}\PY{l+s+s2}{epsilon\PYZus{}1\PYZdl{} which is in direct contradiction with \PYZdl{}x }\PY{l+s+s2}{\PYZbs{}}\PY{l+s+s2}{leq y + }\PY{l+s+s2}{\PYZbs{}}\PY{l+s+s2}{epsilon\PYZdl{}}
          
          \PY{l+s+s2}{As an example for \PYZdl{}}\PY{l+s+s2}{\PYZbs{}}\PY{l+s+s2}{epsilon\PYZdl{}, we can use \PYZdl{}(x\PYZhy{}y)/}\PY{l+s+si}{\PYZob{}2\PYZcb{}}\PY{l+s+s2}{\PYZdl{}.}
          \PY{l+s+s2}{\PYZdq{}\PYZdq{}\PYZdq{}}
          \PY{n}{display}\PY{p}{(}\PY{n}{Markdown}\PY{p}{(}\PY{n}{text}\PY{p}{)}\PY{p}{)}
\end{Verbatim}


    Proof by contradiction:

Assume \(x > y\) then we have:

\(\exists \epsilon_1 > 0: x >= y + \epsilon_1\) which is in direct
contradiction with \(x \leq y + \epsilon\)

As an example for \(\epsilon\), we can use \((x-y)/{2}\).

    
    \begin{Verbatim}[commandchars=\\\{\}]
{\color{incolor}In [{\color{incolor}233}]:} \PY{n}{text} \PY{o}{=} \PY{n}{ps}\PY{p}{[}\PY{l+m+mf}{2.31}\PY{p}{]}\PY{o}{=} \PY{l+s+sa}{r}\PY{l+s+s2}{\PYZdq{}\PYZdq{}\PYZdq{}}
          \PY{l+s+s2}{We have 2 cases:}
          
          \PY{l+s+s2}{Case 1: \PYZdl{}n\PYZdl{} is even which should be easy to prove.}
          
          \PY{l+s+s2}{Case 2: \PYZdl{}n\PYZdl{} is odd: \PYZdl{}n = 2k+1 }\PY{l+s+s2}{\PYZbs{}}\PY{l+s+s2}{rightarrow n\PYZca{}3 + n = (8k\PYZca{}3 + 12k\PYZca{}2 + 6k+1) + (2k+1) = }
          \PY{l+s+s2}{8k\PYZca{}3 + 12k\PYZca{}2+8k + 2 = 2(4k\PYZca{}3+6k\PYZca{}2+4k+1) = 2m\PYZdl{}}
          \PY{l+s+s2}{\PYZdq{}\PYZdq{}\PYZdq{}}
          \PY{n}{display}\PY{p}{(}\PY{n}{Markdown}\PY{p}{(}\PY{n}{text}\PY{p}{)}\PY{p}{)}
\end{Verbatim}


    We have 2 cases:

Case 1: \(n\) is even which should be easy to prove.

Case 2: \(n\) is odd:
\(n = 2k+1 \rightarrow n^3 + n = (8k^3 + 12k^2 + 6k+1) + (2k+1) = 8k^3 + 12k^2+8k + 2 = 2(4k^3+6k^2+4k+1) = 2m\)

    
    \begin{Verbatim}[commandchars=\\\{\}]
{\color{incolor}In [{\color{incolor}234}]:} \PY{n}{text} \PY{o}{=} \PY{n}{ps}\PY{p}{[}\PY{l+m+mf}{2.42}\PY{p}{]}\PY{o}{=} \PY{l+s+sa}{r}\PY{l+s+s2}{\PYZdq{}\PYZdq{}\PYZdq{}}
          \PY{l+s+s2}{\PYZdl{}}\PY{l+s+s2}{\PYZbs{}}\PY{l+s+s2}{forall n }\PY{l+s+s2}{\PYZbs{}}\PY{l+s+s2}{in Z: n\PYZdl{} is odd \PYZdl{}}\PY{l+s+s2}{\PYZbs{}}\PY{l+s+s2}{iff n+2\PYZdl{} is odd.}
          
          
          \PY{l+s+s2}{First we prove }
          \PY{l+s+s2}{\PYZdl{}}\PY{l+s+s2}{\PYZbs{}}\PY{l+s+s2}{forall n }\PY{l+s+s2}{\PYZbs{}}\PY{l+s+s2}{in Z: n\PYZdl{} is odd \PYZdl{}}\PY{l+s+s2}{\PYZbs{}}\PY{l+s+s2}{rightarrow n+2\PYZdl{} is odd:}
          
          \PY{l+s+s2}{\PYZdl{}n = 2k+1 }\PY{l+s+s2}{\PYZbs{}}\PY{l+s+s2}{rightarrow n+2 = 2k(even)+3(odd) }\PY{l+s+s2}{\PYZbs{}}\PY{l+s+s2}{rightarrow n+2\PYZdl{} is odd}
          
          
          \PY{l+s+s2}{Second we prove }
          \PY{l+s+s2}{\PYZdl{}}\PY{l+s+s2}{\PYZbs{}}\PY{l+s+s2}{forall n }\PY{l+s+s2}{\PYZbs{}}\PY{l+s+s2}{in Z: n+2\PYZdl{} is odd \PYZdl{}}\PY{l+s+s2}{\PYZbs{}}\PY{l+s+s2}{rightarrow n\PYZdl{} is odd:}
          
          \PY{l+s+s2}{\PYZdl{}n+2 = 2k+1 }\PY{l+s+s2}{\PYZbs{}}\PY{l+s+s2}{rightarrow n = 2k+1\PYZhy{}2 = 2k\PYZhy{}1 }\PY{l+s+s2}{\PYZbs{}}\PY{l+s+s2}{rightarrow n = 2k(odd) + (\PYZhy{}1)(odd) \PYZhy{}\PYZgt{} \PYZdl{} n is odd}
          \PY{l+s+s2}{\PYZdq{}\PYZdq{}\PYZdq{}}\PY{o}{.}\PY{n}{replace}\PY{p}{(}\PY{l+s+s1}{\PYZsq{}}\PY{l+s+s1}{\PYZhy{}\PYZgt{}}\PY{l+s+s1}{\PYZsq{}}\PY{p}{,} \PY{l+s+s2}{\PYZdq{}}\PY{l+s+se}{\PYZbs{}\PYZbs{}}\PY{l+s+s2}{rightarrow}\PY{l+s+s2}{\PYZdq{}}\PY{p}{)}
          \PY{n}{display}\PY{p}{(}\PY{n}{Markdown}\PY{p}{(}\PY{n}{text}\PY{p}{)}\PY{p}{)}
\end{Verbatim}


    \(\forall n \in Z: n\) is odd \(\iff n+2\) is odd.

First we prove \(\forall n \in Z: n\) is odd \(\rightarrow n+2\) is odd:

\(n = 2k+1 \rightarrow n+2 = 2k(even)+3(odd) \rightarrow n+2\) is odd

Second we prove \(\forall n \in Z: n+2\) is odd \(\rightarrow n\) is
odd:

\$n+2 = 2k+1 \rightarrow n = 2k+1-2 = 2k-1 \rightarrow n = 2k(odd) +
(-1)(odd) \rightarrow \$ n is odd

    
    \begin{Verbatim}[commandchars=\\\{\}]
{\color{incolor}In [{\color{incolor}235}]:} \PY{n}{text} \PY{o}{=} \PY{n}{ps}\PY{p}{[}\PY{l+m+mf}{2.42}\PY{p}{]}\PY{o}{=} \PY{l+s+sa}{r}\PY{l+s+s2}{\PYZdq{}\PYZdq{}\PYZdq{}}
          \PY{l+s+s2}{\PYZdl{}}\PY{l+s+s2}{\PYZbs{}}\PY{l+s+s2}{forall n }\PY{l+s+s2}{\PYZbs{}}\PY{l+s+s2}{in Z: n\PYZdl{} is odd \PYZdl{}}\PY{l+s+s2}{\PYZbs{}}\PY{l+s+s2}{iff n+2\PYZdl{} is odd.}
          
          
          \PY{l+s+s2}{First we prove }
          \PY{l+s+s2}{\PYZdl{}}\PY{l+s+s2}{\PYZbs{}}\PY{l+s+s2}{forall n }\PY{l+s+s2}{\PYZbs{}}\PY{l+s+s2}{in Z: n\PYZdl{} is odd \PYZdl{}}\PY{l+s+s2}{\PYZbs{}}\PY{l+s+s2}{rightarrow n+2\PYZdl{} is odd:}
          
          \PY{l+s+s2}{\PYZdl{}n = 2k+1 }\PY{l+s+s2}{\PYZbs{}}\PY{l+s+s2}{rightarrow n+2 = 2k(even)+3(odd) }\PY{l+s+s2}{\PYZbs{}}\PY{l+s+s2}{rightarrow n+2\PYZdl{} is odd}
          
          
          \PY{l+s+s2}{Second we prove }
          \PY{l+s+s2}{\PYZdl{}}\PY{l+s+s2}{\PYZbs{}}\PY{l+s+s2}{forall n }\PY{l+s+s2}{\PYZbs{}}\PY{l+s+s2}{in Z: n+2\PYZdl{} is odd \PYZdl{}}\PY{l+s+s2}{\PYZbs{}}\PY{l+s+s2}{rightarrow n\PYZdl{} is odd:}
          
          \PY{l+s+s2}{\PYZdl{}n+2 = 2k+1 }\PY{l+s+s2}{\PYZbs{}}\PY{l+s+s2}{rightarrow n = 2k+1\PYZhy{}2 = 2k\PYZhy{}1 }\PY{l+s+s2}{\PYZbs{}}\PY{l+s+s2}{rightarrow n = 2k(odd) + (\PYZhy{}1)(odd) \PYZhy{}\PYZgt{} \PYZdl{} n is odd}
          \PY{l+s+s2}{\PYZdq{}\PYZdq{}\PYZdq{}}\PY{o}{.}\PY{n}{replace}\PY{p}{(}\PY{l+s+s1}{\PYZsq{}}\PY{l+s+s1}{\PYZhy{}\PYZgt{}}\PY{l+s+s1}{\PYZsq{}}\PY{p}{,} \PY{l+s+s2}{\PYZdq{}}\PY{l+s+se}{\PYZbs{}\PYZbs{}}\PY{l+s+s2}{rightarrow}\PY{l+s+s2}{\PYZdq{}}\PY{p}{)}
          \PY{n}{display}\PY{p}{(}\PY{n}{Markdown}\PY{p}{(}\PY{n}{text}\PY{p}{)}\PY{p}{)}
\end{Verbatim}


    \(\forall n \in Z: n\) is odd \(\iff n+2\) is odd.

First we prove \(\forall n \in Z: n\) is odd \(\rightarrow n+2\) is odd:

\(n = 2k+1 \rightarrow n+2 = 2k(even)+3(odd) \rightarrow n+2\) is odd

Second we prove \(\forall n \in Z: n+2\) is odd \(\rightarrow n\) is
odd:

\$n+2 = 2k+1 \rightarrow n = 2k+1-2 = 2k-1 \rightarrow n = 2k(odd) +
(-1)(odd) \rightarrow \$ n is odd

    
    \begin{Verbatim}[commandchars=\\\{\}]
{\color{incolor}In [{\color{incolor}236}]:} \PY{k}{for} \PY{n}{key}\PY{p}{,} \PY{n}{val} \PY{o+ow}{in} \PY{n}{ps}\PY{o}{.}\PY{n}{items}\PY{p}{(}\PY{p}{)}\PY{p}{:}
              \PY{n}{display}\PY{p}{(}\PY{n}{Markdown}\PY{p}{(}\PY{l+s+s2}{\PYZdq{}}\PY{l+s+s2}{**}\PY{l+s+si}{\PYZpc{}3.2f}\PY{l+s+s2}{**: }\PY{l+s+si}{\PYZpc{}s}\PY{l+s+s2}{  **[\PYZhy{}1]**}\PY{l+s+s2}{\PYZdq{}}\PY{o}{\PYZpc{}}\PY{p}{(}\PY{n}{key}\PY{p}{,} \PY{n}{val}\PY{p}{)}\PY{p}{)}\PY{p}{)}
\end{Verbatim}


    \textbf{1.08}:
\(m = 2k_1 + 1, n = 2k_2 + 1 \Rightarrow m + n = 2k_1 + 2_k2 + 2 = 2(k_1+k_2+1) = 2k\)
\textbf{{[}-1{]}}

    
    \textbf{1.11}:
\(m = 2k_1 + 1, n = 2k_2 \Rightarrow m \times n = (2k_1 + 1)\times (2_k2) = 2 \times k2(2k_1+1) = 2k\)
\textbf{{[}-1{]}}

    
    \textbf{1.20}:
\(x \in X \rightarrow x \in X \vee x \in Y \rightarrow x \in X \cup y \rightarrow X \subseteq X \cup Y\)
\textbf{{[}-1{]}}

    
    \textbf{1.26}:
\(A \in \wp(X) \cup \wp(Y) -> A \in \wp(X) \vee A \in \wp(Y) \rightarrow A \subseteq X \vee A \subseteq Y \rightarrow A \subseteq X \cup Y \rightarrow A \in \wp(X \cup Y)\)
\textbf{{[}-1{]}}

    
    \textbf{1.45}: (1):
\(a \in X \cup Y \rightarrow a \in X \vee a \in Y \rightarrow a \in Y \vee a \in X \rightarrow a \in Y \cup X \rightarrow X \cup Y \subseteq Y \cup X\)

(2):
\(a \in Y \cup X \rightarrow a \in Y \vee a \in X \rightarrow a \in X \vee a \in Y \rightarrow a \in X \cup Y \rightarrow Y \cup X \subseteq X \cup Y\)

\begin{enumerate}
\def\labelenumi{(\arabic{enumi})}
\item
  \begin{itemize}
  \item
    \begin{enumerate}
    \def\labelenumii{(\arabic{enumii})}
    \setcounter{enumii}{1}
    \tightlist
    \item
      \(\rightarrow X \cup Y = Y \cup X\)
    \end{enumerate}
  \end{itemize}
\end{enumerate}

The same goes for \(\cap\) \textbf{{[}-1{]}}

    
    \textbf{2.02}: Converse: for all \(x \in \mathbb{R}\) if \(x\) is
irrational, then \(x^2\) is irrational.

Proof by contradiction: \(x = \sqrt(2)\) is irrational but \(x^2 = 2\)
is rational. \textbf{{[}-1{]}}

    
    \textbf{2.14}: Since it says `'There exists'', it is sufficient to find
a single example that satisfies the argument.

Example:
\(a = 2, b = 1/2 \rightarrow a\)\textsuperscript{\(b = a\)}\((1/2) = \sqrt{2}\)
\textbf{{[}-1{]}}

    
    \textbf{2.15}: Proof by contradiction:

Assume \(x > y\) then we have:

\(\exists \epsilon_1 > 0: x >= y + \epsilon_1\) which is in direct
contradiction with \(x \leq y + \epsilon\)

As an example for \(\epsilon\), we can use \((x-y)/{2}\).
\textbf{{[}-1{]}}

    
    \textbf{2.31}: We have 2 cases:

Case 1: \(n\) is even which should be easy to prove.

Case 2: \(n\) is odd:
\(n = 2k+1 \rightarrow n^3 + n = (8k^3 + 12k^2 + 6k+1) + (2k+1) = 8k^3 + 12k^2+8k + 2 = 2(4k^3+6k^2+4k+1) = 2m\)
\textbf{{[}-1{]}}

    
    \textbf{2.42}: \(\forall n \in Z: n\) is odd \(\iff n+2\) is odd.

First we prove \(\forall n \in Z: n\) is odd \(\rightarrow n+2\) is odd:

\(n = 2k+1 \rightarrow n+2 = 2k(even)+3(odd) \rightarrow n+2\) is odd

Second we prove \(\forall n \in Z: n+2\) is odd \(\rightarrow n\) is
odd:

\$n+2 = 2k+1 \rightarrow n = 2k+1-2 = 2k-1 \rightarrow n = 2k(odd) +
(-1)(odd) \rightarrow \$ n is odd \textbf{{[}-1{]}}

    

    % Add a bibliography block to the postdoc
    
    
    
    \end{document}
